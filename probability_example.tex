\documentclass{article}
\usepackage{amsmath}
\usepackage[document]{ragged2e}

\DeclareMathOperator{\arctg}{\int\int}

\begin{document}
Each van has 7 seats. So, there are 14 numbered seats in two vans. \hfill \break

The total number of ways in which 3 girls and 9 boys can sit on these seats is 
\(^{14}C_{12} \times 12 \). \hfill \break

So, total number of seating arrangements = \(^{14}C_{12} \times 12 \) \hfill \break

In a van 3 girls can choose adjacent seats in the back row in two ways 
(1, 2, 3, or 2, 3, 4). So, the number of ways in which 3 girls can sit in the back row
on adjacent seats is \(2 (3!)\) ways. The number of ways in which 9 boys can sit 
on remaining 11 seats is \(^{11}C_{9} \times 9! \) ways. Therefore, the number
of ways in which 3 girls and 9 boys can sit in two vans
\[ = 2(3!) \times ^{11}C_{9} \times 9! + 2(3!) \times ^{11}C_{9} \times 9! + ^{11}C_{9} \times 9!
= 4 \times 3! \times ^{11}C_{9} \times 9!
\]
Hence, required probability \[=\frac{4 \times 3! \times ^{11}C_{9} \times 9!}{^{14}C_{12} \times 12!}
= \frac{1}{91} \]
 
\end{document}  