\documentclass{article}
\usepackage{amsmath}
\DeclareMathOperator{\arctg}{\int\int}

\begin{document}
The binomial coefficient is defined by the next expression:
When displaying fractions in-line, for example \(\frac{3x}{2}\) 
you can set a different display style: 
\( \displaystyle \frac{3x}{2} \).

This is also true the other way around

\[ f(x)=\frac{P(x)}{Q(x)} \ \ \textrm{and} 
\ \ f(x)=\textstyle\frac{P(x)}{Q(x)} \] \\\\

 Integral \(\int_{a}^{b} x^2 \,dx\) inside text	\\

(i) \(\int x^4 \, dx\) = \(\frac{x^{4 + 1}}{4 + 1}\) + C = \(\frac{x^5}{5}\) + C \\\\




Testing notation for limits
\[
    \lim_{h \to 0 } \frac{f(x+h)-f(x)}{h}
.\]
This operator changes when used alongside 
text \( \lim_{h \to 0} (x-h) \). \\

User-defined operator for arctangent:
\[
    \arctg \frac{\pi}{3} = \sqrt{3}
.\]

\[ \sqrt{a}=b \qquad \text{if} \quad b^2=a \] \\

(ii) \(\int \sqrt{x}\, dx\) = \(\int x^\frac{1}{2}\, dx\) = \(\frac{x^\frac{1}{2} + 1}{\frac{1}{2} + 1}\) + C = \(\frac{2}{3}x^\frac{3}{2}\) + C

(iii) \(\int \frac{2}{1 - \cos{2x}} \, dx \) = \(\int \frac{2}{2\cos^2{x}} \, dx\) = \(\int \sec^2{x} \, dx\) = \(\tan{x} + C\)

(iv) \(\int \frac{\cos{2x} + 2\sin^2{x}}{\cos^2{x}} \, dx = \)

\end{document}